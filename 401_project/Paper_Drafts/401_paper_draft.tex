\documentclass[12]{article}

\usepackage[margin=1in]{geometry}
\usepackage{setspace}
\usepackage{graphicx}
\usepackage{verbatim}
\usepackage{amsmath, amsfonts, amssymb}
\usepackage{dsfont}
\usepackage[none]{hyphenat}
\usepackage{float}

\setlength{\parskip}{\baselineskip}

\begin{document}

\begin{titlepage}
\begin{center}
\Huge
Jim Vargas Paper \\
Title Page \\
\today
\end{center}
\end{titlepage}

\tableofcontents 
\thispagestyle{empty}
\clearpage \setcounter{page}{1}


	\section{Introduction}
	
	BACKGROUND ON PROBLEM
	
	\begin{align*}
\Delta u(z) = 0,\enspace z\in \Omega &&
u(z)=h(z),\enspace z\in \Gamma
	\end{align*}
where $\Delta$ is the Laplacian operator, $\Delta = \nabla^2 = \left( \frac{\partial^2}{\partial x^2} + \frac{\partial^2}{\partial y^2}\right)$.
	\begin{align*}
u(z)&\approx \mathrm{Re}[r(z)] \\
r(z) &= \underbrace{\sum_{j=1}^{N_1} \frac{a_j}{z-z_j}}_\text{"Newman"} + \underbrace{\sum_{j=0}^{N_2} b_j (z-z_*)^j}_\text{"Runge"} 
	\end{align*}
	
	WHY RATIONAL FUNCTIONS
	
	USES FOR THE LAPLACE OPERATOR
	
	BARYCENTRIC COORDINATES?
	
	
	\section{Preliminaries}
	We start with a few definitions and theorems from Complex Analysis. These will help us understand and analyze our approximation $r$. 
	
	\textbf{Definition 1.} A function $f$ is called \textit{holomorphic} if, when it is defined in some $\epsilon$-neighborhood of $z_0$, then it is differentiable in some $\epsilon$-neighborhood of $z_0$. 
	
	We note that this definition requires complex-differentiation, which is very similar to but not identical to real-differentiability. The real and imaginary parts of any holomorphic function are harmonic, meaning they satisfy Laplace's equation, which can be seen in the following theorem. 
	
	\textbf{Theorem 1. Cauchy-Riemann Conditions for Differentiability.} A function $f=u+i\,v$ is differentiable at a point $z_0=x_0+i\,y_0$ if and only if the partial derivatives of $u$ and $v$ satisfy
	\begin{align*}
u_x(x_0,y_0)=v_y(x_0,y_0) &&
u_y	(x_0,y_0)=-v_x(x_0,y_0),
	\end{align*}
where we use the standard notation $\frac{\partial u}{\partial x}=u_x$, and so on. Furthermore, $f'(z_0)=u_x(x_0,y_0)+i\, v_x(x_0,y_0)=u_y	(x_0,y_0)-i\, v_y(x_0,y_0)$.

	This tells us that $u_x=v_y$ and $u_y=-v_x$. If we assume we can differentiate once more and that the second partial derivatives of $u$ and $v$ exist and are continuous, then we can obtain the following identity by taking derivatives of these equalities and adding:
	\begin{align*}
\Delta u=u_{xx}+u_{yy}=v_{yx}-v_{yx}=0.
	\end{align*}
The same can be said for $v$ in a similar way. Holomorphic and harmonic functions have many nice properties, so establishing a connection between our problem and these concepts is advantageous. One such property is the \textit{Maximum Principle}, which states that a harmonic function on a closed domain achieves a maximum on the boundary of the domain, and this will guarantee us an identifiable error bound on our approximation.

	HOLOMORPHIC=ANALYTIC
	
	ROOT EXPONENTIAL CONVERGENCE
	
	INTEGRAL FORMULA
	
		
	\section{Main Theorems}
	Theorems from Gopal and Trefethen.
	
	\textbf{Theorem 3.} Let $f$ be a bounded analytic function in the slit disk $A_\pi$ that satisfies $f(z)=O(|z|^\delta)$ as $z \to 0$ for some $\delta > 0$, and let $\theta \in (0,\pi /2)$ be fixed. Then for some $0< \rho < 1$ depending on $\theta$ but not on $f$, there exist type $(n-1,n)$ rational functions $\{r_n\}$, $1 \leq n < \infty$, such that
	\begin{align*}
||f-r_n||_\Omega = O(e^{-C \sqrt{n}})
	\end{align*}
as $n \to \infty $ for some $C>0$, where $\Omega = \rho A_\theta$. Moreover, each $r_n$ can be taken to have simple poles only at
	\begin{align*}
\beta_j = -e^{-\sigma j/\sqrt{n}}, \enspace 0\leq j \leq n-1,
	\end{align*}
where $\sigma >0$ is arbitrary.

	\textbf{Theorem 4.} Let $\Omega$ be a convex polygon with corners $w_1 , \ldots , w_m$, and let $f$ be an analytic function in $\Omega$ that is analytic on the interior of each side segment and can be analytically continued to a disk near each $w_k$ with a slit along the exterior bisector there. Assume $f$ satisfies $f(z)-f(w_k)=O(|z-w_k|^\delta)$ as $z \to w_k$ for each $k$ for some $\delta >0$. There exist degree $n$ rational functions $\{r_n\},\, 1 \leq n < \infty$ such that
	\begin{align*}
||f-r_n||_\Omega=O(e^{-C\sqrt{n}})
	\end{align*}
as $n\to \infty$ for some $C>0$. Moreover, each $r_n$ can be taken to have finite poles only at points exponentially clustered along the exterior bisectors at the corners, with arbitrary clustering parameter $\sigma$, as long as the number of poles near each $w_k$ grows at least in proportion to $n$ as $n\to \infty$.

	
	\section{Numerical Applications and Experiments}
	
	Present the algorithm probably, lots of figures...



\pagebreak
\bibliographystyle{acm}
\bibliography{401_bib}

\end{document}
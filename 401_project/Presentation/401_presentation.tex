\documentclass{seminar}

% From Doc.
\usepackage{amssymb,amsmath}
%\usepackage{ulem}
\usepackage{semlcmss}
%\usepackage{scicomp}
\usepackage{epsfig}
\usepackage{framed}
\usepackage[T1]{fontenc}
\usepackage{multirow}
\usepackage{bm}

% From me
\usepackage{multicol}
\usepackage{ dsfont }


\def\lfoot{Portland State Applied and Computational Math Group} 
\def\rfoot{Michigan Tech University,  Oct. 4 2019}



\begin{document} 
\pagestyle{headings}
%\slidepagestyle{mypagestyle}
\centerslidesfalse

\begin{slide} %%% Title
\begin{center}
Jim's yucky talk \\
wowee
\end{center}
\end{slide} %%% Title




\begin{slide} %%% 1
\large This talk is based on...\small
\begin{center}
	\includegraphics[scale=0.2]{./PNG/llogo}\\
	%\includegraphics[scale=0.5]{./PNG/paper_abstract}
	\small
	\emph{Solving Laplace Problems with Corner Singularities via Rational Functions}
\end{center}

\begin{itemize}
	\item ...A paper written by Gopal and Trefethen, published in SIAM Journal on Numerical Analysis September 2019
	\item The Lightning Laplace code, based on the paper, yields accurate approximations quickly (on nice problems)
	\item https://epubs.siam.org/doi/pdf/10.1137/19M125947X
	\item https://people.maths.ox.ac.uk/trefethen/lightning.html
\end{itemize}
\end{slide} %%% 1




\begin{slide} %%% 2a
\large Here's the problem\\

\small
We wish to find a (real) function $u$ over a domain $\Omega$ (the complex 2-D plane) which satisfies
\begin{align*}
\Delta u(z)=0, \quad z\in \Omega &&
u(z)=h(z), \quad z\in \Gamma .
\end{align*}
In particular, we want to be able to handle a domain with sharp corners, curves etc. 

We will find $r$, and approximation of $u$ ($u\approx\mathrm{Re}[r]$).
\begin{align*}
r(z)= \sum_{j=1}^{N_1} \frac{a_j}{z-z_j} + \sum_{j=0}^{N_2} b_j (z-z_*)^j
\end{align*}
\end{slide} %%% 2a

\begin{slide} %%% 2b
\large Here's the problem\\

\small
We wish to find a (real) function $u$ over a domain $\Omega$ (the complex 2-D plane) which satisfies
\begin{align*}
\Delta u(z)=0, \quad z\in \Omega &&
u(z)=h(z), \quad z\in \Gamma .
\end{align*}
In particular, we want to be able to handle a domain with sharp corners, curves etc. 

We will find $r$, and approximation of $u$ ($u\approx\mathrm{Re}[r]$).
\begin{align*}
r(z)= \sum_{j=1}^{N_1} \frac{a_j}{z-z_j} + \sum_{j=0}^{N_2} b_j (z-z_*)^j
\end{align*}
\includegraphics[scale=.45]{./PNG/Howell_1994} $\quad$
\includegraphics[scale=.4]{./PNG/snowflake}  $\quad$
\includegraphics[scale=.4]{./PNG/randpoly1}
\end{slide} %%% 2b




\begin{slide} %%% 3
\large Why this problem?\\

\small
\begin{itemize}
	\item Problems involving the Laplace operator $\Delta={\nabla}^2$ frequently appear in physical equations:
	\begin{itemize}
		\item Heat Equation
		$\alpha {\nabla}^2 u={\partial}_t u$
		\item Schrodinger Equation
		$\left[ \frac{-{\hbar}^2}{2m}{\nabla}^2 + V \right]\Psi=i \hbar \, \partial_t \Psi$
		\item Wave Equation
		$c^2 {\nabla}^2 u={\partial}_t^2 u$
		\item And more...
	\end{itemize}
	\item Functions which satisfy Laplace's Equation have very nice properties, and are called harmonic.
	
\end{itemize}
\end{slide} %%% 3




\begin{slide} %%% 4a
\large Some nice properties of functions of interest \\

\small
\begin{itemize}
	\item The real and imaginary parts of a holomorphic (and thus also an analytic) function $f=u+iv$ are harmonic;
	\item $f$ is also smooth (infinitely differentiable); by extension this applies to $u$ and $v$ as well.
	\item Maximum Principle: a harmonic function on a compact domain attains a max. (and min.) on the boundary.
\end{itemize}
\end{slide} %%% 4a

\begin{slide} %%% 4b
\large Some nice properties of functions of interest \\

\small
\begin{itemize}
	\item The real and imaginary parts of a holomorphic (and thus also an analytic) function $f=u+iv$ are harmonic;
	\item $f$ is also smooth (infinitely differentiable); by extension this applies to $u$ and $v$ as well.
	\item Maximum Principle: a harmonic function on a compact domain attains a max. (and min.) on the boundary.
\end{itemize}

On a simply connected domain we can construct a holomorphic function from a harmonic one: given $u$, define $g=u-iu$. The theory will work with harmonic functions, which will trickle down to our problem.

If $r$ approximates $f$, having real part $u$, the worst we'll do over the whole domain in approximating $u$ is $||u(z)-\mathrm{Re}[r(z)]||$, $z \in \Gamma$.
\end{slide} %%% 4b




\begin{slide} %%% 5
\large Back to the problem \\
\small
\begin{align*}
r(z) = \underbrace{\sum_{j=1}^{N_1} \frac{a_j}{z-z_j}}_\text{"Newman"} + \underbrace{\sum_{j=0}^{N_2} b_j (z-z_*)^j}_\text{"Runge"}
\end{align*}

\begin{itemize}
	\item Using the scheme in the paper, we can have root exponentially good approximations for $u$. The task at hand is finding the coefficients $a_j$, $b_j$.
	\begin{align*}
	||f-r_n||_\Omega = O(e^{-C\sqrt{n}})
	\end{align*}
	\item The theorems in the paper are based on interpolation, showing existence.
	\item In the code, the problem is solved via a least squares approach using QR factorization. Code is written in MATLAB.
\end{itemize}
\end{slide} %%% 5




\begin{slide} %%% 6a
\large Describing $r$ 
\small
\begin{align*}
r(z)= \sum_{j=1}^{N_1} \frac{a_j}{z-z_j} + \sum_{j=0}^{N_2} b_j (z-z_*)^j
\end{align*}
The Newman Part: built to handle corners.

\begin{multicols}{2}
\begin{itemize}
	\item The terms $z_j$ are poles, exponentially clustered near a corner on the exterior of $\Omega$ (works for spacing scaled at least $O(n^{-1/2})$).
	\item "Rational functions are more powerful than polynomials for approximating functions near singularities..."
	\footnote{Lloyd N. Trefethen. 2013. Approximation theory and approximation practice, Society for Industrial and Applied Mathematics.}
\end{itemize}
\includegraphics[scale=4]{./PNG/corner_nodes_illust}
\end{multicols}
\end{slide} %%% 6a

\begin{slide} %%% 6b
\large Describing $r$
\small
\begin{align*}
r(z)= \sum_{j=1}^{N_1} \frac{a_j}{z-z_j} + \sum_{j=0}^{N_2} b_j (z-z_*)^j
\end{align*}
The Runge part: built to handle the interior.
\begin{multicols}{2}
\begin{itemize}
	\item The term $z_*$ is an expansion point, near the middle of $\Omega$.
	\item Polynomials can approximate root exponentially well on a nice domain (going back to Runge).
\end{itemize}
\includegraphics[scale=4]{./PNG/corner_nodes_illust}
\end{multicols}
\end{slide} %%% 6b




\begin{slide} %%% 7a
\large The function $r$ is harmonic 
\small
\begin{align*}
r(z)= \sum_{j=1}^{N_1} \frac{a_j}{z-z_j} + \sum_{j=0}^{N_2} b_j (z-z_*)^j
\end{align*}
To prove $r$ is harmonic, consider $f(z)=1/z$ and $g(z)=z^k$.
The function $f$ can be decomposed as $f=u+iv$, where
\begin{align*}
u(x,y)=\frac{x}{x^2+y^2} &&
v(x,y)=\frac{-y}{x^2+y^2}.
\end{align*}
Taking derivatives will show that $u$ and $v$ satisfy the Cauchy-Riemann equations, ${\partial}_x u={\partial}_y v$, ${\partial}_y u=-{\partial}_x v$.
\end{slide} %%% 7a

\begin{slide} %%% 7b
\large The function $r$ is harmonic 
\small
\begin{align*}
r(z)= \sum_{j=1}^{N_1} \frac{a_j}{z-z_j} + \sum_{j=0}^{N_2} b_j (z-z_*)^j
\end{align*}
To prove $r$ is harmonic, consider $f(z)=1/z$ and $g(z)=z^k$.
The function $f$ can be decomposed as $f=u+iv$, where
\begin{align*}
u(x,y)=\frac{x}{x^2+y^2} &&
v(x,y)=\frac{-y}{x^2+y^2}.
\end{align*}
Taking derivatives will show that $u$ and $v$ satisfy the Cauchy-Riemann equations, ${\partial}_x u={\partial}_y v$, ${\partial}_y u=-{\partial}_x v$, meaning $f$ is harmonic.

Writing $g$ in polar form, then in terms of sines and cosines is enough to see $g$ is harmonic:
\begin{align*}
g(z)=\rho e^{i k \theta} = \rho \cos{(k\theta)} + i \sin{k\theta} .
\end{align*}
Adding these templates, applying translations and scaling as necessary give us our result.
\end{slide} %%% 7b




\begin{slide} %%% 8
\large An important lemma \\

\small
Hermite integral formula for rational interpolation.

Let $\Omega$ be a simply connected domain in $\mathds{C}$ bounded by a closed curve $\Gamma$, and let $f$ be analytic in that domain and extend continuously to the boundary. Let interpolation points ${\alpha}_0, \ldots ,{\alpha}_{n-1} \in \Omega$ and poles ${\beta}_0, \ldots ,{\beta}_{n-1}$ anywhere in the complex plane be given. Let $r$ be the unique type $(n-1,n)$ rational function with simple poles at $\{{\beta}_j\}$ that interpolate $f$ at $\{{\alpha}_j\}$. Then for any $z \in \Omega$,
\begin{align*}
f(z)-r(z)=\frac{1}{2 \pi i} \int_{\Gamma} \frac{\phi (z)}{\phi (t)} \frac{f(t)}{t-z}dt, \\
\phi (z) = \left. \prod_{j=0}^{n-1}(z-{\alpha}_j) \middle/ \prod_{j=0}^{n-1}(z-{\beta}_j). \right.
\end{align*}
\end{slide} %%% 8




\begin{slide} %%% 9
\large First Theorem \\

\small
Let $f$ be a bounded analytic function in the slit disk $A_\pi$ that satisfies $f(z)=O(|z|^\delta)$ as $z \to 0$ for some $\delta > 0$, and let $\theta \in (0,\pi /2)$ be fixed. Then for some $0< \rho < 1$ depending on $\theta$ but not on $f$, there exist type $(n-1,n)$ rational functions $\{r_n\}$, $1 \leq n < \infty$, such that
	\begin{align*}
	||f-r_n||_\Omega = O(e^{-C \sqrt{n}})
	\end{align*}
as $n \to \infty $ for some $C>0$, where $\Omega = \rho A_\theta$. Moreover, each $r_n$ can be taken to have simple poles only at
	\begin{align*}
	\beta_j = -e^{-\sigma j/\sqrt{n}}, \enspace 0\leq j \leq n-1,
	\end{align*}
where $\sigma >0$ is arbitrary.
\end{slide} %%% 9




\begin{slide} %%% 10
\begin{center}
\includegraphics[scale=0.8]{./PNG/A_theta_illust}
\end{center}

\begin{align*}
A_\theta=\{z \in \mathds{C} : |z|<1,\enspace |\mathrm{arg}(z)|<\theta \} &&
\Omega = \rho A_\theta
\end{align*}
\end{slide} %%% 10




\begin{slide} %%% 11
\large Second Theorem \\

\small
Let $\Omega$ be a convex polygon with corners $w_1 , \ldots , w_m$, and let $f$ be an analytic function in $\Omega$ that is analytic on the interior of each side segment and can be analytically continued to a disk near each $w_k$ with a slit along the exterior bisector there. Assume $f$ satisfies $f(z)-f(w_k)=O(|z-w_k|^\delta)$ as $z \to w_k$ for each $k$ for some $\delta >0$. There exist degree $n$ rational functions $\{r_n\},\, 1 \leq n < \infty$ such that
	\begin{align*}
	||f-r_n||_\Omega=O(e^{-C\sqrt{n}})
	\end{align*}
as $n\to \infty$ for some $C>0$. Moreover, each $r_n$ can be taken to have finite poles only at points exponentially clustered along the exterior bisectors at the corners, with arbitrary clustering parameter $\sigma$, as long as the number of poles near each $w_k$ grows at least in proportion to $n$ as $n\to \infty$.
\end{slide} %%% 11



\end{document}